\subsection{Initial Setup \& Basic Exploration}\label{subsec:steps-1-3:-initial-setup-&-basic-exploration}
Since a Linux environment is needed to explore the \texttt{/proc} filesystem
and for POSIX compliance, the first step was to install a Virtual Machine (VM)
and a Linux ISO to run on it.
I already had an installation of Debian set up on my PC, so I used this as my testing environment.\\
I then used basic CLI tools to explore the \texttt{/proc} filesystem.
\begin{figure}[H]
    \centering
    \includegraphics[width=0.7\textwidth]{../../screenshots/step1-lsproc}
    \caption{Exploring \texttt{/proc} using \texttt{ls}}
    \label{fig:step1-lsproc}
\end{figure}
\begin{figure}[H]
    \centering
    \includegraphics[width=0.7\textwidth]{../../screenshots/step2-3-catproc}
    \caption{Further exploration of \texttt{/proc} using \texttt{cat}}
    \label{fig:step2-3-catproc}
\end{figure}
\afterfloat
In \textit{Figure 1}, the CLI output of \texttt{ls /proc} can be observed.
This command lists all the contents of the \texttt{/proc} directory to \texttt{stdout}.
Each folder with a number contains information about a process referenced by its Process ID (PID)\@.
There are other miscellaneous files which contain critical system info which can be observed.\\
In \textit{Figure 2}, two such files are examined, \texttt{version} and \texttt{cpuinfo}, using \texttt{cat}.
Here, information is displayed onto the CLI about what version of \texttt{proc} is being used
and information about the CPU\@.
For example, I allocated 8 of my 16 threads in my Ryzen 7 2700,
and we can observe the system seeing 8 CPU cores available.

\subsection{Finding Process Info}\label{subsec:finding-process-info}
The goal of the next activity was to find the PID of the shell and explore its process status.
We used \texttt{bash} as the shell and \texttt{ps} to get the PID\@.
\begin{figure}[H]
    \centering
    \includegraphics[width=0.7\textwidth]{../../screenshots/step4-ps}
    \caption{Finding the PID of the shell and getting its status}
    \label{fig:step4-ps}
\end{figure}
\begin{figure}[H]
        \centering
        \includegraphics[width=0.7\textwidth]{../../screenshots/step5-status}
        \caption{Viewing the output file}
        \label{fig:step5-status}
    \end{figure}
\afterfloat
In \textit{Figure 3}, the shell PID is obtained using \texttt{ps}.
We then output the contents of \texttt{/proc/PID/status}, redirecting it into status.txt.
The full contents of this file can be found in \texttt{logs/status.log},
however, a screenshot is provided in \textit{Figure 4}.
From this figure, we can see that the process was sleeping at the
time of execution of \texttt{cat} along with many other details.

\subsection{Preparing Experiments \& Understanding Tools}\label{subsec:preparing-experiments-&-understanding-tools}
The goal of this section is to explore the state changes of processes using specially created programs.
Using the provided \texttt{lab1.tar}, we will launch and observe other, more interesting programs.
\begin{figure}[H]
    \centering
    \includegraphics[width=0.7\textwidth]{../../screenshots/step6-tar}
    \caption{Extracting lab1 tar file}
    \label{fig:step6-extract-tar}
\end{figure}

\afterfloat
After extracting the contents of \texttt{lab1.tar},
I moved everything to my workspace directory and reorganized the
project structure for neatness.
Inside the \texttt{lab1} directory, there are three compiled binaries:
\texttt{procmon}, \texttt{calcloop}, and \texttt{cploop}.
\subsubsection{Analysis of Binaries}
\noindent
The purpose of the \texttt{procmon} program is to monitor the
status of a process given its PID\@.
The program accepts an argument \texttt{PID} which it then monitors
using the \texttt{proc} filesystem.\\

\noindent
\texttt{cploop} is a program that creates a file named \textit{fromfile}
containing 500 thousand \texttt{"x"} characters and runs ten iterations of a loop that
sleeps for 10 seconds and copies the contents of \textit{fromfile} to a new
file named \textit{tofile}.\\

\noindent
\texttt{calcloop} is a program that runs 10 iterations of a loop that sleeps
for 3 seconds then increments a variable 400 million times

\subsection{Running Provided Scripts}\label{subsec:running-provided-scripts}
The next step is to run the scripts \texttt{tstcalc} and \texttt{tstcp} provided
in the \texttt{lab1} directory.
I moved the scripts to a \texttt{scripts} subdirectory and slightly modified the
scripts to output the log files to a \texttt{logs} subdirectory for cleanliness.
\begin{figure}[H]
    \centering
    \includegraphics[width=0.7\textwidth]{../../screenshots/step7-scripts}
    \caption{Running tstcalc and tstcp}
    \label{fig:step7-scripts}
\end{figure}
\begin{figure}[H]
    \centering
    \includegraphics[width=0.7\textwidth]{../../screenshots/step7-ls}
    \caption{Confirming generation of \texttt{log} files using \texttt{ls}}
    \label{fig:step7-ls}
\end{figure}
\noindent
I forgot to take screenshots of the verification that the \texttt{log} files
were generated using \texttt{ls}, so I had to do it after the fact using a cloned
version of \texttt{lab1} on my Macbook Air, which is why the filepaths are not consistent.

\subsection{Programming mon.c}\label{subsec:programming-mon.c}
The program \texttt{mon.c} is intended to take the name of another program as an argument.
Using this argument, it launches the other program and monitors it using procmon.
The first challenge of this spec was to convert the relative path
passed in through \texttt{argv} to an absolute path accessible by \texttt{execlp}.

\subsection{Compiling mon.c}\label{subsec:compiling-mon.c}
When it comes to compiling from source in C, I believe it is almost
always worth the effort to create a \texttt{Makefile}
instead of manually calling \texttt{gcc} everytime you want to compile a binary.
I also took the time to reorganize the project structure for cleanliness.

\begin{figure}[H]
    \centering
    \includegraphics[width=0.7\textwidth]{../../screenshots/step10-makefile}
    \caption{Creating a Makefile to automate compilation process}
    \label{fig:step10-makefile}
\end{figure}

\begin{figure}[H]
    \centering
    \includegraphics[width=0.7\textwidth]{../../screenshots/step10-compiling}
    \caption{Running Makefile to recompile binaries}
    \label{fig:step10-compiling}
\end{figure}

\noindent
As demonstrated in \textit{Figure 9}, the result is the same as manually
calling \texttt{gcc -o bin/mon src/mon.c} except it is cleaner and more
organized.

